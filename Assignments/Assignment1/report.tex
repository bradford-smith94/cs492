% Bradford Smith (bsmith8)
% CS 492 Assignment 1 report.tex
% 03/04/2016
% "I pledge my honor that I have abided by the Stevens Honor System."
% ===================================================================

% global document styles =======================================================
\documentclass[11pt, letterpaper]{article}
\usepackage[letterpaper, margin=0.5in]{geometry}
\usepackage[utf8]{inputenc}
\usepackage[T1]{fontenc}
\usepackage{tgbonum}
\usepackage{textcomp}
\usepackage{xcolor} %for colored code formatting
\pagestyle{empty}
\setlength{\tabcolsep}{0em}


% custom macros ================================================================
% simple inline code formatting (using typewriter font)
%\newcommand{\code}[1]{\texttt{#1}}

% colored inline code formatting
\definecolor{codebg}{RGB}{225,225,225}
\definecolor{codefg}{RGB}{200,40,90}
\newcommand{\code}[1]{\colorbox{codebg}{\textcolor{codefg}{\texttt{#1}}}}


% document begins here =========================================================
\begin{document}
\noindent{Bradford Smith (bsmith8)}\\
\noindent{CS 492 Assignment 1 report.pdf}\\
\noindent{\today}\\
\noindent{\textit{``I pledge my honor that I have abided by the Stevens Honor System.''}}\\

\bigskip
\noindent{\Large Performance Analysis}

The performance of your scheduling algorithms can be measured by the total time taken to complete all of the jobs and by the time taken to complete individual jobs. Use the \textit{gettimeofday} or \textit{clock} functions to obtain a timestamp when products are generated by producers and are printed out by consumers. Calculate the time taken to finish consumption by taking the difference between these two timestamps.

For each set of input parameters and variations on scheduling algorithms, calculate the following:
\begin{itemize}
    \item Total time for processing all products
    \item Min, max and average \textit{turn-around} times
    \item Min, max and average \textit{wait} times
    \item Producer throughput
    \item Consumer throughput
\end{itemize}

The \textit{turn-around} time is the time between when a product is produced until when it has been completely consumed (printed out). The \textit{wait} time is the time the product spends waiting in the consumer queue (time between when a product is inserted into the queue and when it is removed for consumption). The \textit{producer throughput} is the number of products inserted into the queue per \textbf{minute}. The \textit{consumer throughput} is the number of products printed out per minute. Note that if the producer throughput is low, it will cause the consumer throughput to be low.

From the above metrics, what can you conclude about the different scheduling algorithms. Which algorithm is better, in terms of producer throughput, consumer throughput, etc.? What differences are there in average times when few products (for example, 100) are generated versus many products (for example, 5000)?
\end{document}

